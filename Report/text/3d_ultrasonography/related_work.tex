Previous work in the field of 3D ultrasound reconstruction include novel reconstruction algorithms and methods, as well as designed and implemented production systems. The field of ultrasound and volume visualization is extensive in itself. Nelson \textit{et al.}\ \cite{nelson1993} present approaches to visualization of 3D ultrasound data including simple ray casting and MPR slice visualization, and Ludvigsen \textit{et al.}\ \cite{ludvigsen2010} describe GPU-based volume ray casting with the Nvidia OptiX library. The rest of this section will focus on work in 3D ultrasound reconstruction. In Appendix \ref{chapter:annotated_bib}, an annotated bibliography of selected references can be found.

\subsection{Categorization of Reconstruction Methods}

The vast number of existing reconstruction approaches have been reviewed and categorized. Fenster \textit{et al.}\ \cite{fenster1996} present several approaches in 3D ultrasound imaging not limited to how to just reconstruct, but also the acquisition of input data and how to render the reconstructed results. An attempt at grouping existing reconstruction algorithms is done by Rohling \textit{et al.}\ \cite{rohling1999}. Their categorization of the methods is broadly defined as either \textit{voxel nearest-neighbor interpolation}, \textit{pixel nearest-neighbor interpolation} or \textit{distance-weighted interpolation}. Solberg \textit{et al.}\ \cite{solberg2007} present a different grouping that take additional types of algorithms into account and has a clearer separation between categories. Their grouping is into \textit{voxel-based}, \textit{pixel-based} and \textit{function-based} methods, and their work also includes a thorough comparison of the algorithms based on performance and quality.

\subsection{Implemented Reconstruction Systems}

Trobaugh \textit{et al.}\ \cite{trobaugh1994} present a formative description of a system where optical tracking is used to orient a freehand ultrasound probe, and includes volume reconstruction by both a voxel-based and a pixel-based method. A different approach is described by Prager \textit{et al.}\ \cite{prager1998}. Their \textit{Stradx} system does not reconstruct a volume, but instead generates MPR slices directly from the ultrasound b-scans. Welch \textit{et al.}\ \cite{welch2000} describes a volume reconstructing system under development that allows for updates to the volume during scanning and also simultaneous visualization at \emph{near} real-time performance. Another attempt presented by Gobbi \textit{et al.}\ \cite{gobbi2002} consists of an implemented system that does simultaneous real-time 3D ultrasound reconstruction and visualization, but is limited to the simple PNN method and only orthogonal MPR slice visualization. Furthermore, the visualization is not synchronous with the reconstruction, and is updated at a lower non-real-time framerate.

\subsection{Reconstruction Algorithms}

3D ultrasound reconstruction is often a trade-off between performance and quality. Trobaugh \textit{et al.}\ \cite{trobaugh1994} and Gobbi \textit{et al.}\ \cite{gobbi2002} use the simple PNN method to enable high performance. The alternative voxel-nearest-neighbor is used by Sherebrin \textit{et al.}\ \cite{sherebrin1996} in their 3D ultrasound system. Barry \textit{et al.}\ \cite{barry1997} use a more sophisticated pixel-based method with an inverse distance weighting kernel around inserted pixels. Different approaches are described by Rohling \textit{et al.}\ \cite{rohling1999} and Sanches \textit{et al.}\ \cite{sanches2000} that fall into the function-based category according the terminology of Solberg \textit{et al.}\ \cite{solberg2007}. Rohling \textit{et al.}\ use splines to construct a volume from the input b-scans, and Sanches \textit{et al.}\ use statistical methods to estimate a function for the interpolation. A recent voxel-based method is described by Coupe \textit{et al.}\ \cite{coupe2005} that takes the probe trajectory into account to improve reconstruction quality, especially for sparse input where there is much space between the b-scans. A performance increasing scheme for fast slice selection is described by Wein \textit{et al.}\ \cite{wein2006} and benefits voxel-based reconstruction methods. Karamalis \textit{et al.}\ \cite{karamalis2009} describe a high performing hybrid reconstruction method partially implemented using GPU texture interpolation features. Huang \textit{et al.}\ \cite{huang2005} describe a technique for utilizing the Fourier domain to take redundant frequency components into account, preserving the high frequencies and resulting in better resolutions.